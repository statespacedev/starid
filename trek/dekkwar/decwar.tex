\documentstyle[12pt,a4,psfig,german]{article}
\pagestyle{empty}
\headheight0cm
\headsep0cm
\textwidth15cm
\textheight24cm
\parindent0cm
\def\3{\ss}
\begin{document}

\begin{center}

\Huge

{\bf DEKKWAR \ \ }

\ \\

\normalsize

\ \\
{\tt Entwicklungsversion nach 1.5.2 f\"ur Ubuntu 10.04 - 27. Dezember 2011}
\vspace{1.0cm} 

\psfig{figure=UFP-logo.ps,width=7cm,height=6.8cm,clip=t,bbllx=130bp,bblly=285bp,bburx=460bp,bbury=575bp}

\vspace{-6.97cm} \hspace{6.9cm}
\psfig{figure=Klingon-logo.ps,width=7cm,height=8.32cm,clip=t,bbllx=160bp,bblly=230bp,bburx=450bp,bbury=570bp}

\vspace{-0.8cm}
\psfig{figure=Romulan-logo.ps,width=14.62cm,height=7cm,clip=t,bbllx=-20bp,bblly=150bp,bburx=620bp,bbury=650bp}

\vspace{1.0cm}

\Large

Anleitung  \ \ \ \ \\

\vspace{1.0cm}

\small

\copyright 1992--2010 by Thies Meincke \ \ \ \ \ \\

\end{center}

\normalsize

\newpage

\section{Einleitung}

Bei {\bf DEKKWAR} k\"ampfen zwei Parteien von Spielern (die {\sl F\"ODERATION}
und das {\sl IMPERIUM}) um die Herrschaft im All. Als St\"orenfried gibt es 
noch den (computergelenkten) {\sl ROMULANER}. 

Das Programmpaket besteht aus mehreren Teilen:

\begin{enumerate}
\item dw\_shm\_inst.c (shm\_inst) - Anlegen des Shared Memory-Segments f\"ur DECWAR
\item dw\_shm\_rem.c (shm\_rem) - Entfernen des Shared Memory-Segments
\item init\_gal.c (init\_gal) - Anlegen des Universums
\item defeat.c (def) - Automatische Steuerung f\"ur Forts, Basen, Satelliten
\item romulan.c (rom) - Der Romulaner
\item dwmain.c (dwmain) - Das Programm f\"ur jeden Spieler.
\end{enumerate}

dwmain wiederum erzeugt Tochterprozesse f\"ur den Scout, verschossene
Torpedos, Flugbomben, Robotschiffe,den Flug durchs Universum und Explosionen, 
die dann unabh\"angig von dwmain laufen. Diese Programme sind:

\begin{enumerate}
\item cockpit.c (cockpit) - Cockpitdarstellung; l\"auft permanent.
\item explosion.c (explosion) - Darstellung eines explodierenden Objekts.
\item move.c (move) - Flug mit Warpantrieb; l\"auft permanent.
\item phaser.c (phaser) - Schu\3 mit Phaser;  l\"auft permanent.
\item missile.c (missile) - laufende Flugbombe;   l\"auft bei Bedarf.
\item robot.c (robot) - aktives Robotschiff;   l\"auft bei Bedarf.
\item scout.c (scout) - aktiver Scout;  l\"auft bei Bedarf.
\item torpedo.c (torpedo) - fliegender Torpedo;  l\"auft bei Bedarf.
\item rtorpedo.c (torpedo) - fliegender Torpedo eines Robotschiffs; 
       l\"auft bei Bedarf.
\end{enumerate}

\section{Einstieg}

Man mu\3 nach dem Starten des Programms dwmain seinen Spielernamen 
(maximal 10 Zeichen, keine Leerzeichen) eingeben. Der jeweilige
Rang ergibt sich aus der bislang erreichten Punktzahl:


\begin{center}
\begin{tabular}{|l|l|}
\hline
Punktzahl & Rang \\ \hline \hline
0 - 14999 & Commander \\ \hline
15000 - 29999 & Captain \\ \hline
30000 - 99999 & Commodore \\ \hline
ab 100000     & Admiral \\ \hline
\end{tabular}
\end{center}

Je nach Rang kann man
sich ein Schiff ausssuchen, das durch den 
Anfangsbuchstaben gekennzeichnet ist. Man mu\3 Captain sein, um einen Kreuzer
kommandieren zu k\"onnen, und mindestens Kommodore f\"ur ein Schlachtschiff.
Auch sonst hat der Rang Einflu\3.

Folgende Schiffe stehen zur Auswahl:


\footnotesize
\hspace{-1.0cm}
\begin{verbatim}
_______________FEDERATION_____________|________________EMPIRE_________________
    Federal Battleship ENTERPRISE (E) |    Imperial Battleship MANTA      (M)
    Federal Battleship LEXINGTON  (L) |    Imperial Battleship PYTHON     (P)
    Federal Battleship YORKTOWN   (Y) |    Imperial Battleship JACKAL     (J)
    Federal Battleship DEFIANT    (D) |    Imperial Battleship VIPER      (V)
    Federal Cruiser    California (C) |    Imperial Cruiser    Falcon     (F)
    Federal Cruiser    Oklahoma   (O) |    Imperial Cruiser    Iscariot   (I)
    Federal Cruiser    Kansas     (K) |    Imperial Cruiser    Bloodhunt  (B)
    Federal Cruiser    Tennessee  (T) |    Imperial Cruiser    Warlord    (W)
    Federal Cruiser    Nevada     (N) |    Imperial Cruiser    Scorpion   (S)
    Federal Cruiser    Utah       (U) |    Imperial Cruiser    Quarrel    (Q)
    Federal Destroyer  kittyhawk  (k) |    Imperial Destroyer  vulture    (v)
    Federal Destroyer  nostromo   (n) |    Imperial Destroyer  wasp       (w)
    Federal Destroyer  discovery  (d) |    Imperial Destroyer  buzzard    (b)
    Federal Destroyer  leonow     (l) |    Imperial Destroyer  hornet     (h)
    Federal Destroyer  endeavour  (e) |    Imperial Destroyer  falchion   (f)
    Federal Destroyer  ticonderoga(t) |    Imperial Destroyer  shark      (s)
    Federal Destroyer  columbia   (c) |    Imperial Destroyer  mamba      (m)
    Federal Destroyer  orion      (o) |    Imperial Destroyer  panther    (p)
    Federal Destroyer  avenger    (a) |    Imperial Destroyer  intruder   (i)
    Federal Destroyer  yorikke    (y) |    Imperial Destroyer  jaguar     (j)
------------------------------------------------------------------------------
\end{verbatim}

\normalsize

Durch die Wahl seines Schiffes legt man sich f\"ur eine der beiden Parteien
fest. Weitere Details zu den Schiffen sind dem Abschnitte Schiffe zu entnehmen.


Ziel des Spiels ist es nat\"urlich, dem Gegner soviele Sch\"aden 
zuzuf\"ugen wie
m\"oglich. Dazu sind die Schiffe bewaffnet. Alle Schiffe verf\"ugen \"uber
Torpedos, Kreuzer und Schlachtschiffe auch \"uber Phaser. Was man im 
einzelnen tun
kann, ist aus der weiter unten aufgef\"uhrten Beschreibung aller
Kommandos zu entnehmen,
die immer \"uber die Tastatur eingegeben werden.

Die Ressourcen, die man zur Verf\"ugung hat, sind beschr\"ankt. Jede Handlung
kostet Energie und andere Vorr\"ate, so da\3 die Vorr\"ate von Zeit zu Zeit
durch Docken an eine Basis oder die Heimatwelt
der eigenen Seite aufgef\"ullt werden m\"ussen.

Durch Beschu\3 oder Kollisionen kann das eigene Schiff besch\"adigt werden,
so da\3 die Funktion einzelner oder mehrer Subsysteme beeintr\"achtigt werden
kann. Wenn man eine bestimmte Menge an Sch\"aden \"uberschritten hat, gilt
das Schiff als zerst\"ort und man fliegt aus dem Spiel. Das gleiche passiert,
wenn man seine Energie verbraucht hat. Einmal zerst\"orte Schiffe sind aus dem
Spiel.

Fast alles, was man macht bringt Punkte - eventuell aber auch negative 
(z.B. das Schie\3en auf eine eigene Basis)!


\section{Die Heimatwelten )F( und ]E[}

Jede Partei verf\"ugt \"uber eine eigene Heimatwelt, von der aus man in
das Spiel einsteigt und an die man auch wieder 
von den benachbarten Sektoren aus andocken kann. Das Symbol f\"ur die
Heimatwelt des Imperiums ist ]E[, f\"ur die F\"oderation )F(. Wie eine
Basis feuert eine Heimatwelt automatisch auf Feinde, jedoch ca. 10 Mal 
heftiger. Die Reichweite ist 9 Sektoren. Feindliche Schiffe werden jedoch
bereits ab einem Abstand von 15 Sektoren per Funk gewarnt: 
,,YOU ENTERED A RESTRICTED AREA! IF YOU VALUE YOUR LIFE BE SOMEWHERE ELSE!''.
Das Kommando zum Andocken lautet HW. Nach dem Andocken befindet man sich in
einen speziellen Zustand, der nicht automatisch beendet wird. Es wird das
Flottenhauptquartier mit den zur Zeit in der Heimatwelt angedockten Schiffen
angezeigt. 

Angezeigt werden ferner die Daten der Heimatwelt. Die Anzeige f\"ur Condition
zeigt normalerweise ,,green'', ist die Heimatwelt besch\"adigt 
zeigt sie ,,YELLOW'', erfolgte k\"urzlich ein Angriff, zeigt sie ,,RED''.

Es gibt eine Reihe von speziellen Kommandos, die man nur ausf\"uhren kann, wenn
man an die Heimatwelt gedockt ist:

\begin{description}
\item [CR n] - Call back Robotship n: R\"uckruf des aktiven eigenen 
               Robotschiffs mit der Nummer n (n=a: alle eigenen aktiven
               Robotschiffe). Notwendiger Rang: Admiral.
\item [KR n] - Kill Robotship n: Selbstzerst\"orung des aktiven eigenen 
               Robotschiffs mit der Nummer n (n=a: alle eigenen aktiven
               Robotschiffe). Notwendiger Rang: Admiral.
\item [LM x1 y1 x2 y2 -  LM x1 y1 p] - Launch Missile: Start einer Flugbombe 
                (Missile) zu den Zielkoordinaten
                x1 y1 \"uber die Koordinaten x2 y2 oder (p) \"uber einen vorher
                aufgezeichneten Pfad (Kommando MP).  
                Man mu\3 Admiral sein und an der Heimatwelt gedockt sein, 
                um eine Flugbombe starten zu k\"onnen.
\item [LS x y] - Launch Scout: Start eines Scoutschiffs 
                zu den Zielkoordinaten
                x y. Man mu\3 mindestens Commodore sein, um einen Scout
                starten zu k\"onnen.
\item [LR n m x y] - Launch Robotship: Start eines Robotschiffs; x,y = Ziel,
                n=Nr. des Robotschiffs, m=Mission (siehe unten).
                Man mu\3 Admiral sein, um ein Robotschiff starten zu k\"onnen.
\item [SD bp] - Set Display bases+planets: Anzeige von Ort und Status der
                eigenen Basen, Planeten, Forts und Satelliten.
\item [SD fl] - Set Display fleet: Anzeige des eigenen Flottenstatus.
\item [SD hw] - Set Display homeworld: Anzeige des Flottenhauptquartiers.
\item [SD m] - Set Display map: \"Ubersichtskarte des Universums mit den
                eigenen Basen, Planeten, Forts und Satelliten.
\item [SD re] - Set Display reconnaissance data: Darstellung aller Sichtungen
                feindlicher Schiffe durch eigene Basen, Planeten, Forts 
                und Satelliten mit Angabe der Position und der Sternzeit, wann
                das Schiff gesichtet wurde. Man mu\3 zumindest Commodore sein,
                um diese Informationen abrufen zu k\"onnen. Sichtungen des 
                Romulaners werden auch der Gegenseite mitgeteilt.
\item [SD ro] - Set Display Robotships: Status\"ubersicht der Robotschiffe.
\item [SElect ?] - Auswahl einen neuen Schiffs mit dem Buchstaben ?. Auf diese
                  Weise kann man das Schiff wechseln, ohne das Spiel beenden 
                  zu m\"ussen. 
\item [STart] - Auslaufen zu einer neuen Mission.
\end{description}

{\sf Abbildung 2: Das ''Cockpit'' eines Raumschiffs bei Spielbeginn und 
                  beim Docken an der Heimatwelt (bei SD hw). Das ? bei
                  der Anzeige von Area bedeutet, da\3 die Basenverteidigung
                  def nicht l\"auft. Bei laufender Basenverteidigung wird die
                  von eigenen Basen, Forts und Satelliten \"uberwachte Raumfl\"ache
                  in $\%$ der Gesamtfl\"ache angezeigt. Ferner wird die Anzahl
                  der Spieler beider Seiten angezeigt (Players: 0/1 bedeutet,
                  das 0 Spieler auf Seiten der F\"oderation spielen, einer auf Seiten
                  des Imperiums). Des weiteren wird links eine \"Ubersicht der 
                  Flotte und der Status der Heimatwelt angezeigt. Infos zu
                  gespeicherten Pfaden der Flugbomben sind ebenfalls dort zu
                  finden.}

\footnotesize
\hspace{-1.0cm}
\begin{verbatim}
 #################        ################## |     -- Imperial HOMEWORLD --    |
 #                                         # | Starfleet Base and Head Quarter |
 #         #####################           # +----------Player-Status----------|
 #-<v                                   0>-# |  Name......:  KOR               |
 #-<w       ->-#-<       >-#-<          1>-# |  Rank......:  Admiral           |
 #-<b       ;>-#-<       >-#-<          2>-# |  Score.....:  363130            |
 #-<h        >-#-<       >-#-<          3>-# |  Ship......:  Falcon            |
 #-<f        >-#-<F      >-#-<          4>-# |  Class.....:  Cruiser           |
 #-<s        >-#-<I      >-#-<          5>-# |-------Match-Status-Empire-------|
 #-<m       M>-#-<B      >-#-<          6>-# |  Bases......:  10  Players: 0/1 |
 #-<p       P>-#-<W      >-#-<          7>-# |  Planets....:   0   0   0   0   |
 #-<i       J>-#-<S      >-#-<          8>-# |  Forts......:   0               |
 #-<j       V>-#-<Q      >-#-<          9>-# |  Satellites.:   0               |
 ########################################### |  Score......:   0               |
 # Position: 183 077  Stardate: 6568.1     # |  Area.......:   ? (def?)        |
 # Condition: GREEN   Damages:    0 ( 0 %) # |------------Messages-------------|
 # MP no. 0: empty                         # |             Welcome             |
 # MP no. 1: empty                         # |             Welcome             |
 # MP no. 2: length=121, end point 014 044 # |                                 |
 #                                         # |   Initial position: Homeworld   |
 ########################################### |         Status: docked          |
+-------------------------------------------+|  Enter STart for an new mission |
|COMMAND:                                   ||                                 |
+-------------------------------------------++---------------------------------+
\end{verbatim}

\normalsize

\newpage

\section{Das Cockpit}

Wenn man zu einer Mission die Heimatwelt mit dem Kommando STart verlassen
hat, \"andert sich die Bildschirmanzeige. Man erh\"alt voreingestellt ein
Radarbild oben links, wieder die Kommandozeile unten links. Oben rechts
sind Informationen zum Schiff und Kommandanten zu finden, unten rechts
sind wieder aktuelle Meldungen zu finden (siehe auch die folgende Abbildung).

\footnotesize
\hspace{-1.0cm}
\begin{verbatim}
   134 136 138 140 142 144 146 148 150 152    Imperial Destroyer  wasp
196 . . . . . . . . . . . . . . . . . . *196  Comm. Off.: Admiral    KOR
195 . . . . . . . . . . . . . . . . * . .195 |-------------Status--------------|
194 . . . . . . . . . . . . . . . . . . .194 |  Position.....:     143  187    |
193 . . . . * . * . . . . . . . . . . . .193 |  Main Energy..:    3808   95%   |
192 . . . . . . . . . * . . . . . . . . .192 |  Shield Energy:    1500  100%   |
191 . . . . . . . . . . . *   . . . . . .191 |  Phaser Energy:    ----  ---    |
190 . . . . . . . .   . . . . . . . . . .190 |  Fuel (tons)..:     100  100%   |
189 . . . . . . . . . . . . . . . @ . . .189 |  Torpedoes....:      15  100%   |
188 . . * . . . . . . . . . . . . . . . .188 |  Mines........:       6  100%   |
187 . . . . . . . . . w . . . . . . . . .187 |  Satellites...:       2  100%   |
186 . . . . . . . * . . . . . . . * . . .186 |  Damages......:       0    0%   |
185 . . . . . . . . . . . . . . . . . . .185 |  Score........:       0         |
184 . . . . . . . . . . . . . . . . . . .184 |  Stardate.....:  1580.2         |
183 . . . . . . . . . . . . . . . . . . .183 |  SH:- RA:- SV:- MP:-      TT:+  |
182 . . . . . . . . . . . . . . . * . . .182 |------------Messages-------------|
181 . . . . . . . . . . . . . * . . . . .181 |HW: Select w: C+ M+ T- P         |
180 . . . . . . . * . . . . . . . . . . *180 |HW: Congratulation, you are now  |
179 . . . . . . . . . . . . . . . . . . .179 |  the commanding officer of the  |
178 . . . . . . . . . . . . . . . . . . .178 |Imperial Destroyer wasp          |
   134 136 138 140 142 144 146 148 150 152   |IN: Total:   8 bases detected    |
|--------------------------------------------|IN: Next base at 114 147         |
|COMMAND:                                    |IN: 2nd next base at 90 102      |
|------------------------------------------------------------------------------|
\end{verbatim}

\normalsize

In der Kommandozeile unten links im Cockpit k\"onnen \"uber die 
Tastatur Befehle eingegeben werde. Diese werden im Abschnitt Befehle
detailliert erl\"autert.

\section{Das Universum}

Das Spielfeld (Universum) besteht aus 
$200\times200$ Sektoren. Dort kann man eine Vielzahl verschiedene
Objekte vorfinden, die im folgenden aufgef\"uhrt werden:

\begin{description}
\item [.] - Freies Feld: Diese Felder sind zum Flug geeignet.
\item [ ] - Schwarzes Loch: Wer in ein Schwarzes Loch fliegt, wird mit einer
            bestimmten Wahrscheinlichkeit vernichtet. Alternativ kommt
            er an irgendeiner Stelle des Universums wieder zum Vorschein. 
            Der Navigationscomputer kann Kollisionen mit Schwarzen 
            L\"ochern nicht verhindern.
\item[@] - Planet: Ein Planet ist anfangs neutral und kann f\"ur die jeweilige
                 Seite erobert werden. Dann schie\3en selbstt\"atig auf
                 feindliche Schiffe. Planeten haben vier Ausbaustufen. 
                 Ihre Feuerkraft und Reichweite betr\"agt je nach Ausbaustufe
                 Ausbaustufe $\times$ 100 Einheiten und 
                 Ausbaustufe $\times$ 2 Sektoren.
                 Ein Planet der Ausbaustufe 4 kann zu einer Basis 
                 ausgebaut werden.
\item [$\ast$] - Stern: Durch Beschu\3 oder Rammen kann ein Stern explodieren.
                 Dabei werden benachbarte Objekte besch\"adigt oder vernichtet.
                 Es kann zu Kettenreaktion kommen, wenn Sterne nebeneinander
                 liegen.
\item[\#] - Fort des Imperiums: schie\3t selbstst\"andig auf feindliche Schiffe, 
            die Reichweite betr\"agt 9 Sektoren.
            Meldungen \"uber Sichtungen von Feindschiffen an die Heimatwelt.
            Meldungen \"uber Angriffe gehen an alle eigenen Schiffe.
\item[\$] - Fort der F\"oderation: schie\3t selbstst\"andig 
            auf feindliche Schiffe, die Reichweite betr\"agt 9 Sektoren.
            Meldungen \"uber Sichtungen von Feindschiffen an die Heimatwelt.
            Meldungen \"uber Angriffe gehen an alle eigenen Schiffe.
\item[\~] - Satellit der F\"oderation: meldet das Auftauchen nicht 
            getarnter feindlicher Schiffe an alle eigenen Schiffe und die
            Heimatwelt. Die 
            Reichweite betr\"agt 9 Sektoren.
\item[\^] - Satellit des Imperiums: meldet das Auftauchen nicht 
            getarnter feindlicher Schiffe an alle eigenen Schiffe und die
            Heimatwelt. Die 
            Reichweite betr\"agt 9 Sektoren.
\item[$\cdot$] - Mine: Explodiert, wenn man sie rammt.
\item[( )] - Basis der F\"oderation: siehe Abschnitt Basen.
\item[[ ]] - Basis des Imperiums: siehe Abschnitt Basen.
\item[=] - Gaswolke: Sie kann nur mit Impulsantrieb durchflogen werden. 
           Die Schilde sollten eingeschaltet sein, um Sch\"aden zu vermeiden. 
           Durch Gaswolken kann man nur mit Phasern schie\3en. 
           Die Phaserstrahlen werden dabei geschw\"acht. Je nach Einstellung
           des Flags radarlee in init\_gal.c sind Gaswolken f\"ur Radarstrahlen
           durchl\"assig oder nicht.
\item[+] - Torpedo oder Torpedoblindg\"anger: explodiert bei Kollision.
\item[,] - getarntes Schiff, Scoutschiff oder Flugbombe.
\item[-] - Flugbombe (Missile), ist jedoch unbesch\"adigt getarnt.
\item[!] - Robotschiff der F\"oderation.
\item[$|$] - Robotschiff des Imperiums.
\item [\S] - Sprungtor: Ein Schiff kann von einem Sprungtor zum anderen
             springen und so langwierige Fl\"uge vermeiden. Siehe Befehle
             JG und JU.
\item[:] - Scoutschiff der F\"oderation: siehe Abschnitt Scoutschiffe.
\item[;] - Scoutschiff des Imperiums: siehe Abschnitt Scoutschiffe.
\item[R] - Romulaner: siehe Abschnitt Romulaner
\item[]E[] - Heimatwelt des Imperiums: 
               siehe Abschnitt \"uber Heimatwelten.
\item[)F(] - Heimatwelt der F\"oderation: siehe Abschnitt Heimatwelten.
\item[Buchstaben] - Schiffe: sie sind durch den ersten Buchstaben ihres Namens
                    erkennbar (wenn sie nicht getarnt sind!).
\end{description}

\section{Schiffe}

Die Schiffe des Imperiums und der F\"oderation sind in drei Klassen einge-
teilt: Schlachtschiffe (je 4), Kreuzer (je 6) und Zerst\"orer (je 8), wobei
letztere durch Kleinbuchstaben zu erkennen sind. Die drei Klassen haben
unterschiedliche Eigenschaften, innerhalb der Klassen sind die Schiffe
jedoch gleich. Durch Wahl seines Schiffes legt man zugleich fest, ob man zum
{\sl Imperium} oder zur {\sl F\"oderation} geh\"ort.\\

\begin{center}
\begin{tabular}{|l|r|r|r|r|} \hline
\ \ & Schlachtschiff & Kreuzer & Zerst\"orer & Robotschiff\\ \hline
Hauptenergie & 5000 & 4000 & 4000 & 2000 \\
Schildenergie & 2500 & 2000 & 1500 & 1800 \\
Phaserenergie & 2500 & 2000 & - & - \\
Torpedos & 25 & 20 & 15 & 20 \\
Torpedorohre & 3 & 2 & 1 & 1 \\
Minen & 10 & 6 & 6 & - \\
Satelliten & 6 & 3 & 2 & - \\
Impulstreibstoff & 130 t & 120 t & 100 t & 200 t\\
Max. Schadenspunkte & 2500 & 2000 & 1500 & 2000 \\
Maximale Phaserst\"arke & 500 & 400 & - & - \\ 
Mittlere Torpedost\"arke & 250 & 250 & 250 & 250 \\ \hline
\end{tabular} \\
\end{center}

Es gilt  ferner: Je gr\"o\3er das Schiff, desto schneller kann es 
Planeten erobern und auszubauen sowie Forts bauen. Je gr\"o\3er
das Schiff ist, desto langsamer fliegt es. Zerst\"orer verf\"ugen nicht
\"uber Phaser. Nur Schlachtschiffe k\"onnen Sprungtore bauen. 

Ein Zerst\"orer hat 1, ein Kreuzer 2
und ein Schlachtschiff 3 Torpedorohre. Das Nachladen dauert 1,5 jeweils
Sekunden. Ein Schlachtschiff kann also bis zu drei Torpedos sofort
hintereinander feuern, dann dauert es bis zum n\"schsten Schu\3
1,5 Sekunden. Ein Zerst\"orer braucht bereits f\"ur den 2. Schu\3
1,5 Sekunden. Der Status der Torpedorohre l\"a\3t sich im 
Statusfenster des Cockpits erkennen: TT:=?  (?=+ geladen, ?=R 
,,reloading'', ?=- leer).

In der Darstellung des Cockpits (Abschnitt Cockpit) 
sind ferner die jeweiligen aktuellen
Werte f\"ur die vorhandenen Energien, Torpedos usw.  wiedergegeben.

Wird ein Schiff besch\"adigt, kann der Schaden an folgenden Sektionen
auftreten und f\"ur entsprechende Funktionsbeeintr\"achtigungen bzw.
-ausf\"alle sorgen (gilt nicht f\"ur Robotschiffe):

\begin{center}
\begin{enumerate}
\item Warp-Antrieb
\item Impuls-Antrieb
\item Torpedorohre
\item Phaser (falls vorhanden)
\item Schilde
\item Computersysteme
\item Lebenserhaltungssystem
\item Radar
\item Funkanlage
\item Tarnvorrichtung
\end{enumerate}
\end{center}


\section{Liste der Befehle}

In der Kommandozeile unten links im Cockpit
k\"onnen \"uber die Tastatur Befehle eingegeben werde. Jeder Befehl besteht
zumeist aus zwei Buchstaben. Eventuell sind Optionen oder Werte anzugeben.

\begin{description}      
\item [BR] - Build Robotship: Bau eines neuen Robotschiffs. Man mu\3 an die
      heimatwelt gedockt sein und den Rang eines Admirals haben, um diese 
      Kommando ausf\"uhren zu k\"onnen.
\item [BU x y] - Build: Ausbau des Planeten auf x y um eine Stufe. 
        Der Planet mu\3 zur eigenen Partei geh\"oren und direkt neben dem
        Schiff liegen. Die Schilde m\"ussen gesenkt sein . Beispiel: BU 45 124
\item [CA x y] - Capture: Erobern einen feindlichen oder neutralen
        Planeten auf x y. Er mu\3 direkt neben
        dem Schiff liegen. Die Schilde m\"ussen gesenkt sein.
\item [CP n] - Clear (missile) Path n: L\"oschen des Pfades Nr. n f\"ur eine
      Flugbombe.
\item [CR n] - Call back Robotship n: R\"uckruf des aktiven eigenen 
               Robotschiffs mit der Nummer n (n=a: alle eigenen aktiven
               Robotschiffe). Notwendiger Rang: Admiral. Nur bei Dock an die
               Heimatwelt.
\item [DA] - Damages: Auflistung der Sch\"aden in allen Sektionen.
\item [DO] - Dock: Andocken an eine eigene Basis. Die Basis mu\3 direkt neben
    dem Schiff liegen.
\item [FO x y] - Fort: Bau eines Forts auf der Position x y. Dabei 
         m\"ussen x und y 
         eine Position direkt neben dem Schiff bezeichnen, die Schilde m\"ussen
         DOWN sein.
\item [HE] - Help: ohne Parameter: Ausgabe einer Liste aller Kommandos. Mit
            Parameter: Ausgabe einer Info zu dem Kommando, das der Parameter
            kennzeichnet. Beispiel HE DO - Info zum Dock--Kommando.
\item [HO on / off] - Hood: Tarnvorrichtung an/aus. F\"ur HO on 
             m\"ussen die Schilde
             ''DOWN'' sein. Der Schiffkennbuchstabe wird durch '','' ersetzt, 
             feindliche Basen, Forts und Satelliten ''sehen'' einen nicht, 
             feindliche Schiffe k\"onnen nicht computergest\"utzt feuern, man
             kann aber selber nicht schie\3en. Der Romulaner allerdings kann
             auf ein getarntes Schiff schie\3en!
\item [HW] - Homeworld: Docken an die eigene Heimatwelt, siehe auch Abschnitt
          ,,Heimatwelt''.
\item [ID ?] - Identify: Identifikation eines Schiff mit dem Kennbuchstaben
      ?. Beispiel ID e
\item [IM dx dy] - Impuls: Flug mit Impulsantrieb, max. 1 Sektor, pro Sektor 
          1 Tonne Treibstoffverbrauch. Beispiel: IM 1 -1
\item [IN ?] - Info: Information \"uber: ?=P - Planeten; ?=B - Basen; ?
      =F - Forts; ?=H - Heimatwelt; ?=j - Sprungtore ?=s - Scout ?=m 
      Flugbombe (Missile); - ?=r Robotschiffe. Beispiel: IN b. Erweiterung 
      f\"ur Robotschiffe: IN r n f\"ur Robotschiff Nr. n: Detailierte Info.
\item [JG x y code] - Jumpgate: Bau eines Sprungtores auf der Position x y
          (neben dem Schiff). Das Tor erh\"alt die Nummer code (siehe Abschnitt
          Sprungtore).
\item [JU c1 c2] - Jump: Sprung durch das Sprungtor mit der Nummer c1 zum
          Sprungtor mit der Nummer c2 (siehe Abschnitt Sprungtore).
\item [KM] - Kill Missile: die eigene aktive Flugbombe zerst\"oren.
\item [KR n] - Kill Robotship n: Selbstzerst\"orung des aktiven eigenen 
               Robotschiffs mit der Nummer n (n=a: alle eigenen aktiven
               Robotschiffe).  Notwendiger Rang: Admiral. Nur bei Dock an die
               Heimatwelt.
\item [KS] - Kill Scout: Eigenes Scoutschiff vernichten, zumeist um ein neues
          starten zu k\"onnen.
\item [LE X=....] - Learn: Belegung der Taste X mit dem Befehl ... .
      Beispiel: LE X=PH c E 400. Es geht nur f\"ur die Tasten X und Y!
\item [LE Y=....] - Learn: Belegung der Taste Y mit dem Befehl ... .
      Beispiel: LE Y=TO 77 124. Es geht nur f\"ur die Tasten X und Y!
\item [LM x1 y1 x2 y2 -  LM x1 y1 p] - Launch Missile: Start einer Flugbombe 
                (Missile) zu den Zielkoordinaten
                x1 y1 \"uber die Koordinaten x2 y2 oder (p) \"uber einen vorher
                aufgezeichneten Pfad (Kommando MP).  
                Man mu\3 Admiral sein und an der Heimatwelt gedockt sein, 
                um eine Flugbombe starten zu k\"onnen.
\item [LS x y] - Launch Scout: Starten eines Scoutschiffs mit Ziel x y (geht
                 nur, wenn man Kommodore ist und an die Heimatwelt gedockt 
                 hat).
\item [LR n m x y] - Launch Robotship: Start eines Robotschiffs; x,y = Ziel,
                n=Nr. des Robotschiffs, m=Mission (siehe Abschnitt \"uber
                Robotschiffe).
                Man mu\3 Admiral sein, um ein Robotschiff starten zu k\"onnen.
                Es geht nur von der Heimatwelt aus.
\item [MI x y] - Mine: Auslegen einer Mine auf die  
        Position x y. x und y m\"ussen eine
        Position direkt neben dem Schiff bezeichnen.
\item [MP on/off] - Missile Path: Aufzeichnung einen Flugweges einer 
            Flugbombe durch Abfliegen mit einem Schiff bei MP on. Beginn
            bei der eigenen Heimatwelt. Ende der Aufzeichnung mit MP off.
            Anzeige des Status und die
            L\"ange des aufgezeichneten Pfades (maximal MISSILE\_FUEL+1)
            erfolgen im Cockpit und mit dem Kommando IN m.
\item [MR dx dy] - Move Relative: Bewegung mit Warp-Antrieb um 
           dx Sektoren in X-Richtung und dy Sektoren
           in Y-Richtung. Beispiel: MR 2 -4
\item [PH C ? n] - Phaser: Feuern mit Phaser (computerberechnet, vgl. TO C ?),
           n=Schu\3st\"arke, ?=Ziel.
           Ist das Ziel mehr als 4 Sektoren entfernt, so schw\"acht sich
           der Phaserstrahl pro Sektor um ca. 10% ab!). Beispiel: PH c E 400
\item [PH x y n] - Phaser: mit St\"arke n auf die Koordinaten x y abfeuern,
           maximale Reichweite 9 Sektoren, Abschw\"achung wie vorher. 
           Beispiel PH 12 199 200
\item [RA on/off] - Ramming: Bei RA ON k\"onnen Objekte gerammt werden, 
            bei RA OFF
            wird eine Kollision mit den meisten Objekten vom Bordcomputer
            verhindert (Falls er o.k. ist). Das Rammen verursacht Sch\"aden
            auf beiden Seiten. Das Rammen der Heimatwelten ist t\"odlich f\"ur
            das rammende Schiff!
\item [RE n] - Repair: Reparatur der Sektion n; verringert Sch\"aden in der 
       betreffenden Sektion um 20 Einheiten. Anzeige der Sektionsnummern
       durch den Befehl DA. Beispiel RE 8
\item [RS] - Restart: Startet die Prozesse f\"ur Cockpit usw. neu; nur
             f\"ur Entwicklungsarbeiten gedacht, im Spiel normalerweise nicht
             ben\"otigt.
\item [SA x y] - Satellit: Aussetzen eines Beobachtungssatelliten 
        auf die Position
        x y. x und y m\"ussen eine Position direkt neben dem Schiff bezeichen,
        die Schutzschilde m\"ussen DOWN sein.
\item [SC] - Score: Anzeige der aktuellen Punktzahl.
\item [SD x] - Set Display: Anzeigemodi f\"ur das linke obere Fenster. 
               Es gibt folgende Auswahlm\"oglichkeiten x, die z.T. aber nur
               beim Docken an die Heimatwelt zur Verf\"ugung stehen:
      \begin{description}
      \item [SD bp] - Set Display Bases Planets: Anzeige von Ort und Status der
                eigenen Basen, Planeten, Forts und Satelliten.
      \item [SD fl] - Set Display Fleet: Anzeige des eigenen Flottenstatus.
      \item [SD hw] - Set Display Homeworld: Anzeige des Flottenhauptquartiers.
      \item [SD m] - Set Display Map: \"Ubersichtskarte des Universums mit den
                eigenen Basen, Planeten, Forts und Satelliten.
      \item [SD ra] - Set Display Radar: Radarbild anzeigen lassen
                      (Standardanzeige).
      \item [SD re] - Set Display Reconnaissance Data: Darstellung aller 
                Sichtungen
                feindlicher Schiffe durch eigene Basen, Planeten, Forts 
                und Satelliten mit Angabe der Position und der Sternzeit, wann
                das Schiff gesichtet wurde. Man mu\3 zumindest Commodore sein,
                um diese Informationen abrufen zu k\"onnen.
      \item [SD ro] - Set Display Robotships: Status\"ubersicht der Robotschiffe.
      \end{description}
\item [SE ? ...] - Send: Verschicken einer Funknachricht ... an das Ziel ?.
            ?=$\ast$ - an alle Spieler, ?=\& an alle Spieler 
            der eigenen Partei,
            oder bei Eingabe des Schiffskennbuchstaben an das jeweilige
            Schiff (falls im Spiel). Beispiel: SE E Wo bist du?
\item [SElect ?] - Select: Auswahl eines neuen Schiffs ? (geht nur, wenn man
                   an die Heimatwelt angedockt hat).
\item [SH up/down] - Shields: Heben oder Senken der Schutzschilde.
\item [SP n] - Save (missile) Path n: Speichern eines mit ,,MP on'' 
               aufgezeichneten Pfades f\"ur eine Flugbombe. n bezeichnet die 
               Pfadnummer.
\item [STart] - Start: Ausdocken aus der Heimatwelt oder einer Basis.
\item [ST] - Stopp: Stopp des Fluges mit Warp-Antrieb (nach Kommando MR).
\item[SV on/off]: Save Modus: Bei SV on wird kein Torpedo oder Phaser auf
       das angegebene Ziel gefeuert, wenn ein Hindernis im Weg liegt. 
       Defaultwert ist OFF. Anzeige des Status erfolgt im Cockpit.
\item [SY] - Symbols: Erl\"auterung der Symbole im All.
\item [TE ? n] - Transport Energy:  Transportieren von n Energieeinheiten 
           vom eigenen Schiff (Hauptenergie) zum Schiff
           mit dem Kennbuchstaben ? (0-9: Robotschiff). Beispiel TE L 300
\item [TF ? n] - Transport Fuel:  Transportieren von n Tonnen Treibstoff 
           vom eigenen Schiff zum Schiff
           mit dem Kennbuchstaben ? (0-9: Robotschiff). Beispiel TF L 30
\item [TT ? n] - Transport Torpedo:  Transportieren von n Torpedos 
           vom eigenen Schiff zum Schiff
           mit dem Kennbuchstaben ? (0-9: Robotschiff). Beispiel TF m 3
\item [TO C ?] - Torpedo: computerberechnet einen Torpedo
          auf das Schiff ? schie\3en. Beispiel TO c F (r=Romulaner).
\item [TO x y] - Torpedo: Torpedo auf die Koordinaten x y feuern, 
          max. Reichweite 9 Sektoren. Beispiel TO 125 23.
\item [TR ? ? n] - Transfer: Transferieren von n Energieeinheiten 
           von Energiespeicher ?  zu Energiespeicher ?; Energiespeicher:
               m = Hauptenergie, s = Schildenergie p = Phaserenergie.
          Beispiel: TR s m 500
\item [!] Befehlswiederholung, f\"uhrt den vorher eingebenen Befehl nochmal aus
         (Return ist nicht n\"otig!).
\item [QU] - Quit: Ausstieg aus dem Spiel (ohne Return!). Nicht m\"oglich
         im Feuerbereich von feindlichen Schiffen, Forts, Basen
         und der feindlichen Heimatwelt.
\item [X] Ausf\"uhren des mit dem LE-Kommando eingegeben 
          Befehls f\"ur die Taste X (ohne Return!).
\item [Y] Ausf\"uhren des mit dem LE-Kommando eingegeben 
          Befehls f\"ur die Taste Y (ohne Return!).
\end{description}      

Das Bauen von Forts und Basen oder das Aussetzen von Satelliten oder Minen
kostet sowohl Zeit als auch Energie. Manche Kommandos funktionieren nicht oder
nur teilweise, wenn Teile des Schiffs oder beim Docken/Ausdocken die Basis
oder Heimatwelt besch\"adigt sind.


\section{Die Basen ( ) und [ ]}

Wichtige St\"utzpunkte sind eigene Basen. Eine gewisse Anzahl wird
bei der Initialisierung des Universum erzeugt, weitere kann man bauen,
indem man Planeten erobert und sie ausbaut.

Es gibt drei spezielle Kommandos f\"ur die Nutzung von Basen:

\begin{description}
\item [DO] - Dock: Docken an eine Basis. Die Basis mu\3 neben dem Schiff
             liegen.
\item [STart]: - Start: Ausdocken und Start zu einer neuen Mission.
\item [IN b] - Informationen \"uber Basen: Gibt die Anzahl der eigenen
Basen an und zeigt den Status aller Basen im Radarbereich des Schiffes.
\end{description}

Die nachfolgende Abbildung zeigt die Cockpit-Anzeige nach dem Andocken
an eine Basis. Man erkennt vier Docks (Bays). In einem Dock liegt das
eigene Schiff, die drei anderen sind in diesem Fall frei. Es k\"onnen
also maximal vier Schiffe gleichzeitig an eine Basis docken.
Beim Eindocken werden die verbrauchten Vorr\"ate (Energie, Torpedos, Minen,
Satelliten, Treibstoff) aufgef\"ullt und eventuell vorhandene Sch\"aden
repariert, was umso l\"anger dauert, je gr\"o\3er sie sind.

Angezeigt werden ferner die Daten der Basis. Die Anzeige f\"ur Condition
zeigt normalerweise ,,green'', ist die Basis besch\"adigt zeigt sie ,,YELLOW'',
erfolgte k\"urzlich ein Angriff, zeigt sie ,,RED''.

Zu beachten ist, da\3 alle Schiffe, die an eine Basis gedockt sind, zerst\"ort
werden, wenn die Basis zerst\"ort wird. Ferner sind nach dem Ausdocken eines
Schiffes dessen Schilde ,,down''.

\footnotesize

\hspace{-1.0cm}
\begin{verbatim}
###################    ###################    Imperial Battleship  PYTHON
#######                            #######    Comm. Off.: Admiral    KOR
#######                            #######   |-------------Status--------------|
#######----<P                 >----#######   |  Position.....:     126  105    |
#######                            #######   |  Main Energy..:    5000  100%   |
#######                            #######   |  Shield Energy:    2500  100%   |
#######----<                  >----#######   |  Phaser Energy:    2500  100%   |
#######                            #######   |  Fuel (tons)..:     130  100%   |
#######                            #######   |  Torpedoes....:      25  100%   |
##########################################   |  Mines........:      10  100%   |
##########################################   |  Satellites...:       5  100%   |
#+----------Imperial_Starbase-----------+#   |  Damages......:       0    0%   |
#| Position: 126 105  Condition: Green  |#   |  Score........:       0         |
#| Shields : 3000 (100 %) UP            |#   |  Stardate.....:  8397.1         |
#| Damages : 0    (0   %)               |#   |  SH:- RA:- SV:- MP:-    D TT:+++|
#|                                      |#   |------------Messages-------------|
#|                                      |#   |  Enter STart for an new mission |
#|                                      |#   |                                 |
#|                                      |#   |IN: Total:  10 bases detected    |
#+--------------------------------------+#   |IN: Next base at 126 105         |
##########################################   |IN: 2nd next base at 103 149     |
|-------------------------------------------||DO:  DOCKING at BASE!            |
|COMMAND:                                   ||DO:Supplies filled up:E,T,M,S,F  |
+-------------------------------------------||---------------------------------+\end{verbatim}

\normalsize

Die Basen beider Seiten schie\3en automatisch auf Feinde (wenn def l\"auft)
und haben die gleichen Eigenschaften: Sie schie\3en mit Phaser einer
mittleren St\"arke von 500 Einheiten (mit einer Reichweite von
9 Sektoren), haben Schilde mit einer 
Maximalenergie von 3000 Einheiten und vertragen 2500 Scha\-dens\-punkte.
Sie melden alle Sichtungen feindlicher Schiffe an die Heimatwelt. Angriffe
werden an alle aktiven Schiffe der eigenen Flotte gemeldet.

\section{Die Scoutschiffe : und ;}

Jede Partei kann von der eigenen Heimatwelt aus je ein Scoutschiff starten,
das der Fernaufkl\"arung dient, da es nicht von feindlichen Forts und Basen
erkannt wird, denn der Scout fliegt getarnt. Wenn er besch\"adigt wird,
kann diese Tarnung unwirksam werden. Dann wird der Scout auch von 
feindlichen Forts und Basen automatisch beschossen.

Der Start kann jedoch nur von einem Spieler mit dem Rang Commodore oder
Admiral durchgef\"uhrt werden (siehe Abschnitt ,,Heimatwelt''). Der
Scout fliegt automatisch auf die angegebene Zielposition und verbleibt dort.
Jeder Spieler kann sich jedoch das Radarbild des Scouts anzeigen lassen
(s.u.). Da der Scout nur einen
begrenzten Energievorrat mitf\"uhrt, ist auch seine Lebensdauer begrenzt.
Es kann jedoch nur ein Scout jeder Partei aktiv sein. Um einen neuen
Scout in ein anderes Zielgebiet schicken zu k\"onnen, mu\3 der alte
Scout vorher zerst\"ort werden. Er kann nicht zur\"uckgerufen werden.

Kommandos auf das Scoutschiff bezogen:
\begin{description}
\item [LS x y] - Launch Scout: Start des Scoutschiffs zu den Zielkoordinaten
                 x y. Dieses Kommando kann nur ausgef\"uhrt werden, wenn man
                 mindestens Commodore ist und an die Heimatwelt gedockt ist.
\item [IN s] - Info Scout: Informationen zum Status, Ort und Ziel des eigenen
               Scoutschiffs.
\item [SD sc] - Set Display Scout: Anzeige des Radarbildes des Scouts (mit
                Restenergie- und Zielanzeige).
\item [KS] - Kill Scout: Scout wird zerst\"ort um z.B. einen neuen Scout
             starten zu k\"onnen.
\end{description}

\section{Flugbomben (Missiles) -}

Jede Partei kann von der eigenen Heimatwelt aus je eine Flugbombe starten,
Der Start kann jedoch nur von einem Spieler mit dem Rang eines
Admiral durchgef\"uhrt werden. Die
Flugbombe fliegt automatisch zur angegebene Zielposition, und zwar nicht auf 
dem direkten Weg von der Heimatwelt aus, sondern \"uber einen frei w\"ahlbaren
Punkt. Alternativ kann man einen Flugpfad ganz oder teilweise vorgeben.
Dazu fliegt man mit einem Schiff von her Heimatwelt zu einem beliebigen
Zielpunkt. Zur Aufzeichnung mu\3 der Befehl MP on gesetzt werden, am Ende
der Aufzeichnung der Befehl MP off. Der so aufgezeichnet Pfad mu\3 vor einer 
Nutzung allgemein zug\"anglich gemacht werden. Dazu wird er mit dem Kommando
SP n als Nr. n in das Pfadverzeichnis aufgenommen. Ein eventuell existierender
Pfad mit dieser Nummer wird dabei \"uberschrieben. 
Von der Heimatwelt aus kann dann
der Schiffskommandant mit LM x y p n die Flugbombe mit Endziel x y starten,
wobei zun\"achst der aufgezeichnete Flugweg f\"ur den Pfad Nr. n
geflogen wird. Vom Ende des
Flugweges bis zum Endziel fliegt die Bombe dann auf direktem Weg.

Die Reichweite der Flugbombe ist jedoch begrenzt (Wert MISSILE\_FUEL).

Sobald die Flugbombe auf ihrem Weg auf ein Hindernis trifft, explodiert sie.
Die Sprengkraft betr\"agt im Mittel 3000 Energieeinheiten. Eine
Flugbombe kann also sogar
einem Schlachtschiff gef\"ahrlich werden. Sogar Gaswolken werden vernichtet.
Somit besteht nur mit einer Flugbombe die M\"oglichkeit, Gaswolken zu 
beseitigen.
Nur Schwarze L\"ocher k\"onnen von einer Flugbombe nicht zerst\"ort werden.

Erreicht die Flugbombe die angegebenen Zielkkordinaten, ohne auf ein Hindernis
zu treffen, so bleibt sie als Blindg\"anger (+) liegen, mit der gleichen
Sprengkraft wie ein Torpedo-Blindg\"anger. Es kann dann eine neues Flugbombe
gestartet werden.

Wie beim Scout fliegt die Flugbombe getarnt. Wenn sie besch\"adigt wird,
kann diese Tarnung unwirksam werden. Das geschieht, wenn die H\"alfte
der erlaubten Scha\-dens\-punkte \"uberschritten wird.
Dann wird die Flugbombe von 
feindlichen Forts und Basen automatisch beschossen.

Es kann jedoch nur eine Flugbombe jeder Partei aktiv sein. Um einen neuen
Fulgbombe in ein anderes Zielgebiet schicken zu k\"onnen, mu\3 die alte
Flugbombe vorher zerst\"ort werden. Sie kann nicht zur\"uckgerufen werden.

Eine nicht mehr ben\"otigter Pfad kann mit dem Kommando CP n gel\"oscht werden.

Die Anzahl der verf\"ugbaren Pfade kann man mit IN m erkennen. Sie wird auch im
Dockbild der Heimatwelt angezeigt.

Flugbomben-Kommandos:
\begin{description}

\item [CP -] - Clear (missile) Path: L\"oschen des Flugbomben-Pfads Nr. n.

\item [LM x1 y1 x2 y2 -  LM x1 y1 p n] - Launch Missile: Start einer Flugbombe 
                (Missile) zu den Zielkoordinaten
                x1 y1 \"uber die Koordinaten x2 y2 oder (p) \"uber einen vorher
                aufgezeichneten Pfad mit der Nummer n 
                (siehe auch Kommando MP).  
                Man mu\3 Admiral sein und an die Heimatwelt gedockt sein, 
                um eine Flugbombe starten zu k\"onnen.

\item [MP on/off] - Missile Path: Aufzeichnung einen Flugweges einer 
            Flugbombe durch Abfliegen mit einem Schiff bei MP on. Beginn
            bei der eigenen Heimatwelt. Ende der Aufzeichnung mit MP off.
            Anzeige des Status und die
            L\"ange des aufgezeichneten Pfades (maximal MISSILE\_FUEL+1)
            erfolgen im Cockpit und mit IN m. Umspeichern auf einen allgemein
            nutzbaren Pfad Nr. n erfolgt mit dem Kommando SP n.

\item [IN m] - Info Missile: Informationen \"uber Status, Ort, Ziel,
               Zwischenziel, Sch\"aden und Treibstoffvorrat der Flugbombe.
               Auch Angaben zu einem aufgzeichneten Flugpfad werden ausgegeben.

\item [KM] - Kill Missile: Flugbombe wird zerst\"ort um z.B. eine neue
             starten zu k\"onnen.
\item [SP n] - Save (missile) Path n: Speicher des lokal aufgezeichneten
               Flugbomben-Pfads in das Pfad-Register unter der Nummer n.
\end{description}


Bei der Flugbombe ist noch eine Besonderheit zu beachten: Die Explosionswirkung
erstreckt sich nicht nur auf die aktuelle Koordinate, sondern in halbierter
St\"arke auch auf alle direkt angrenzenden Sektoren! Die Explosion eines Sterns
durch eine Flugbombe kann aber nicht zu einer Kettenreaktion benachbarter
Sterne ausarten, wie es bei Torpedo- oder Phaserbeschu\3 sein kann.


\section{Sprungtore \S}

Um langwierige Fl\"uge zu vermeiden, k\"onnen Sprungtore genutzt werden.
Springtore k\"onnen von jeder Partei genutzt werden, egal wer sie gebaut
hat. Jedoch ist die Kenntnis den jeweiligen individuellen Sprungtorcodes
notwendig. Dabei handelt es sich um eine zweistellige Zahl, die beim
Bau zugewiesen wird und deshalb in der Regel nur der bauenden Partei
bekannt ist ist. Die Zahl der Sprungtore ist begrenzt (z.Z. 6). Man kann nicht
mit eingeschalteter Pfadaufzeichung (MP on) oder mit eingeschalteter Tarnung
(HO on) durch ein Sprungtor fliegen.

Kommandos bez\"uglich der Sprungtore:

\begin{description}
\item [IN j] - Ausgabe der Positionen, des Status und des Codes aller
               von der jeweiligen Partei gebauten und noch existierenden
               Sprungtore.
\item [JG x y code] - Bau eines Sprungtores auf der Position x y
          (neben dem Schiff). Das Tor erh\"alt die Nummer code, eine
          zweistellige Zahl.
\item [JU c1 c2] - Sprung durch das Sprungtor mit der Nummer c1 zum
          Sprungtor mit der Nummer c2.
\end{description}

Nur Schlachtschiffe k\"onnen ein Sprungtor bauen.


\section{Robotschiffe ! und $|$}

Als Admiral kann man von der Heimatwelt aus Robotschiffe zu verschiedenen
Missionen starten lassen. Das Kommando lautet LR n m x y.
Es gibt 10 Robotschiffe mit den Nummern n=0 bis n=9. Es gibt die 
Missionstypen m=P,M und J. Das Ziel hat die Koordinaten x y.

In Dockbild der Heimatwelt tauchen die Robotschiffe nicht mit ihrem jeweiligen
Symbol auf, sondern mit ihrer Nummer.

Die Robotschiffe f\"uhren Ihre Missionen selbst\" andig durch.

\begin{description}
\item [Mission p] -  Patrouille: Flug zum Ziel und zur\"uck zur HW. Feindliche
Ziele werden selbstst\"andig bek\"ampft (mit Torpedos, 18 St\"uck).

\item [Mission m] - Minesweep (Minenr\"aumung). Alle Minen in 
Reichweite auf dem Flug
zu den Zielkoordinaten werden vernichtet (solange genug Munition an Bord ist).
Danach R\"uckkehr zur Heimatwelt.

\item [Mission j] - Jumpgate. Alle Sprungtore in 
Reichweite auf dem Flug
zu den Zielkoordinaten werden vernichtet (solange genug Munition an Bord ist).
Danach R\"uckkehr zur Heimatwelt.

\item [Mission c] - Capture. Der Planet auf den angegebenen Zielkoordinaten 
wird erobert. Danach R\"uckkehr zur Heimatwelt.


\end{description}

Mit SD ro (bei Dock an HW) und IN r kann man sich Informationen \"uber 
die eigenen Robotschiffe anzeigen lassen. Details zu einem Robotschiff mit
IN r n (N=Nummer des Robotschiffs).

Robotschiffe fliegen wie Scouts und Flugbomben mit Impulsantrieb, aber im
Gegensatz zu Scouts und Flugbomben verf\"ugen sie nicht \"uber eine 
Tarnvorrichtung.

Ein Robotschiff unterbricht die Mission und geht in einen Warte-Modus
(WAITING), wenn der Treibstoff aufgebraucht ist oder die Energie unter ein
bestimmtes Minimum gesunken ist.

Mit dem Kommando TE n m k\"onnen m Energieeinheiten vom eigenen Schiff zum
Robotschiff Nr. n tranferiert werden. Bedingung: es ist im Warte-Zustand.
Analog dient das Kommando TF zum Transfer von Treibstoff und das Kommando
TT zum Transfer von Torpedos. Nach dem Auff\"ullen von Treibstoff oder Energie
setzt das Robotschiff automatisch seine Mission fort.

Ein zerst\"ortes Robotschiff kann ersetzt werden (Kommando BR). Um dieses
Kommando ausf\"uhren zu k\"onnen, mu\3 man an die Heimatwelt gedockt sein
und den Rang eines Admirals haben.

\section{Der Romulaner R}

Das Programm {\sf rom} startet dem Romulaner. Solange kein Spieler
im Spiel ist, bleibt er inaktiv. Sobald ein Spieler da ist, startet der
Romulaner (an dem Symbol R erkennbar) an einer zuf\"alligen Stelle
des Universums und fliegt dann in kurzen Spr\"ungen zum etwa 
gegen\"uberliegenden Ort. Er feuert auf alle Basen, Forts und
Schiffe, nicht aber auf Satelliten oder Planeten. Der Romulaner nimmt auch
getarnte Schiffe unter Feuer! Romulaner legt keine Minen und verf\"ugt nicht
\"uber Satelliten. Der Romulaner hat keine Torpedorohre, sondern
schie\3t ausschlie\3lich mittels Phaser.

\begin{center}

Die Leistungsdaten des Romulaners:


\begin{tabular}{|c|c|}\hline
Max. Schadenspunkte & 2500 \\ \hline
Schildenergie & 2500 \\ \hline
Mittlere Phaserst\"arke & 400 \\ \hline
\end{tabular}
\end{center}

Wenn der Romulaner den Zielpunkt im Universum erreicht hat oder abgeschossen
wird, startet er sich nach einer Pause erneut. Das Programm {\sf rom} 
kann mittels Cntrl-C abgebrochen werden.

\section{\"Anderungen}

Neu in 1.5.2: Kommandos KR und CR, Bugfixes.

\end{document}

